\documentclass[a4paper,11pt]{article}
\usepackage{sbpo-template}
\usepackage[brazil]{babel}
\usepackage[latin1]{inputenc}
\usepackage{amsmath,amssymb}
\usepackage{url}
\usepackage[square]{natbib}
\usepackage{indentfirst}
\usepackage{fancyhdr}
\usepackage{graphicx}
\pagestyle{fancy}
\fancyhf{}
\fancyhead[C]{\includegraphics[width=\textwidth]{cabecalho_sbpo.png}}
\renewcommand{\headrulewidth}{0pt}
\setlength\headheight{101.0pt}
\addtolength{\textheight}{-101.0pt}
\setlength{\headsep}{-5mm}

\begin{document}

\title{Test-Assignment Problem} 

\maketitle
\thispagestyle{fancy}

\author{
\name{Murilo Goedert}
\institute{Engenharia de Software
UDESC}
\iaddress{Ibirama/SC - Brasil}
\email{murilogoedert@gmail.com}
}

\author{ 
\name{Victor Hugo Grabowski Beltramini}
\institute{Engenharia de Software
UDESC}
\iaddress{Ibirama/SC - Brasil}
\email {vhbeltramini@gmail.com}
}

\vspace{8mm}
\begin{resumo}
A atribui\c c\~ao de testes \'e um problema de otimiza\c c\~ao complexo, o objetivo deste problema \'e encontrar a melhor maneira de 
distribuir um teste dentro de uma sala a fim de diminuir ao m\'aximo a probabilidade de ocorrer a copia de respostas 
entre os aplicadores da mesma. O presente documento tem como objetivo resolver esse problema de otimiza\c c\~ao implementando 
um conjunto de metaheur\'isticas incluindo estrat\'egias construtivas e por modifica\c c\~ao, e ainda realizando um estudo experimental a partir do resultado 
destas implemeta\c c\~oes
 \end{resumo}

\bigskip
\begin{palchaves}
Aloca\c c\~ao de testes, Metaheur\'isticas, Problemas de otimiza\c c\~ao.

\bigskip
\noindent{T\'opicos (indique, em ordem de PRIORIDADE, o(s) t\'opicos(s) de seu artigo) }
\end{palchaves}


\vspace{8mm}

\begin{abstract}
The test-Assignment problem is a complex optimization problem, the main objetive of the problem is to find the best way to distribute a test within 
a room in order to minimize the probability of copying answers among the applicators. The document aims to solve this optimization problem implementing a
set of metaheuristics including constructive strategies and modification, and still and still carrying out an experimental study on the result
of these implements.

\end{abstract}

\bigskip
\begin{keywords}
Test-Assignment. Metaheuristics. Optimization problems.

\bigskip
\noindent{Paper topics (indicate in order of PRIORITY the paper topic(s))}
\end{keywords}

 
\newpage
\section{Introdu\c{c}\~ao} 
 
A partir do memento que temos a aplica\c c\~ao de um teste com um numero x de concorrentes queremos eliminar ao m\'aximo a probabilidade de que possa ocorrer copia de respostas de outros concorrentes 
pois isso estar\'a afetando o resultado final dos melhores classificados, a partir disso que surge o problema de aloca\c c\~ao de de testes. 
A probabilidade e facilidade de que se ocorra esse poss\'ivel roubo de respostas pode se dar a partir de alguns fatores,
sendo eles por exemplo a proximidade entre as carteiras e o qu\~ao parecido e cada uma das provas a cada par de carteiras pode ser, a partir desses fatores pode ser 
definido uma nota a para a facilidade de que ocorra ou essa infra\c c\~ao de copia de respostas entre provas, ou seja o objetivo de o problema \'e minimizar ao m\'aximo
essa possibilidade. 

Para isso devemos come\c car modelando nosso problema, inicialmente podemos representa-lo como sendo um grafo, cada v\'ertice sendo uma carteira e cada arestas ir\'a conter
o um pesso que define a facilidade que ocorra a famosa cola, ou seja um as respostas de um teste sejam copiadas por outra pessoa que esteja fazendo o mesmo teste. Esse pesso como
abordado anteriormente ser\'a definido pela jun\c c\~ao do peso da distancia entre os pares de carteiras e de qu\~ao cada prova \'e parecida entre estes mesmos pares.

Este problema foi introduzido inicialmente Duives~\citep{duives:13}. para melhorar essa distribui\c c\~ao de testes aplicados na Universidade de Bologna. A partir de seu estudo inicial foi demonstrado que o 
problema pode ser classificado como NP-Dif\'icil, e o formulou o problema como um programa quadr\'atico bin\'ario n\~ao convexo.

\section{Submiss\~ao do Texto Completo}

Ap\'os cadastrar o artigo, o autor \'e convidado a carregar para o sistema de submiss\~ao um arquivo 
de termina\-\c c\~ao DOC ou PDF, com o texto completo. 
A primeira p\'agina desse manuscrito deve conter o t\'itulo do artigo coincidindo exatamente com o informado quando do cadastramento. 
Deve, tamb\'em, incluir novamente os resumos e palavras-chave em portugu\^es ou espanhol e ingl\^es, e o(s) t\'opico(s) organizado(s) 
em ordem de prioridade, mas \textbf{n\~ao pode incluir nomes de autores}.

Este manuscrito ser\'a disponibilizado para o exame pelos revisores, que ter\~ao tamb\'em acesso
 \`as informa\c c\~oes do cadastro, exceto as referentes aos nomes e institui\c c\~oes dos autores. 
Uma vez aceito o artigo, os autores ser\~ao chamados a encaminhar \textbf{vers\~ao final} com a p\'agina inicial completa, isto \'e, com autores, institui\c c\~oes, resumo de no m\'aximo 150 palavras, 3 palavras-chave, t\'opicos, \textit{abstract}, \textit{keywords} e \textit{paper topics} .

As p\'aginas deste texto n\~ao devem vir numeradas, tanto no caso de arquivo enviado quando da submiss\~ao quanto no caso do arquivo com a vers\~ao final do artigo aceito. 
A numera\c c\~ao ser\'a feita posteriormente para o conjunto de todos os artigos.
\textbf{Cabe\c calhos e rodap\'es devem ser deixados em branco}.


\section{ Instru\c c\~oes de Formata\c c\~ao}


Os trabalhos completos devem ter \textbf{no m\'aximo 12 p\'aginas}, inclu\'idos neste limite: a primeira p\'agina com resumo, texto, tabelas, gr\'aficos, agradecimentos e refer\^encias.

Os textos devem utilizar p\'aginas de tamanho \textbf{A4} (29,7 x 21,0 cm) com \textbf{margem superior de 3,3 cm, inferior de 2,5 cm e laterais de 2,9 cm}.
 Devem ser escritos em coluna \'unica, com fonte \textbf{\textit{Times New Roman} 11}. 



\section{ Estilo das Cita\c c\~oes}


As cita\c c\~oes no texto devem estar entre colchetes e conter  os \'ultimos sobrenomes dos autores~\citep{silva:99}, no caso de um ou dois autores, e o \'ultimo sobrenome seguido de "et al." no caso de mais de dois autores, seguidos do \textbf{ano da publica\c c\~ao}, como por exemplo,~\citep{anna:06},~\citep{gates:03}, ~\citep{smith:02}, ~\citep{duives:13}, ~\citep{pele:04}, ~\citep{web:16}.
As refer\^encias no final do texto devem estar em ordem alfab\'etica do \'ultimo sobrenome do primeiro autor. 


~\\
\bibliographystyle{sbpo}
\bibliography{exemplo-latex}


\end{document}

