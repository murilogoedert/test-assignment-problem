\documentclass[a4paper,11pt]{article}
\usepackage{sbpo-template}
\usepackage[brazil]{babel}
\usepackage[latin1]{inputenc}
\usepackage{amsmath,amssymb}
\usepackage{url}
\usepackage[square]{natbib}
\usepackage{indentfirst}
\usepackage{fancyhdr}
\usepackage{graphicx}
\pagestyle{fancy}
\fancyhf{}
\fancyhead[C]{\includegraphics[width=\textwidth]{cabecalho_sbpo.png}}
\renewcommand{\headrulewidth}{0pt}
\setlength\headheight{101.0pt}
\addtolength{\textheight}{-101.0pt}
\setlength{\headsep}{-5mm}

\begin{document}

\title{Test-Assignment Problem} 

\maketitle
\thispagestyle{fancy}

\author{
\name{Murilo Goedert}
\institute{Engenharia de Software
UDESC}
\iaddress{Ibirama/SC - Brasil}
\email{murilogoedert@gmail.com}
}

\author{ 
\name{Victor Hugo Grabowski Beltramini}
\institute{Engenharia de Software
UDESC}
\iaddress{Ibirama/SC - Brasil}
\email {vhbeltramini@gmail.com}
}

\vspace{8mm}
\begin{resumo}
A atribui\c c\~ao de testes \'e um problema de otimiza\c c\~ao complexo, o objetivo deste problema \'e encontrar a melhor maneira de 
distribuir um teste dentro de uma sala a fim de diminuir ao m\'aximo a probabilidade de ocorrer a copia de respostas 
entre os aplicadores da mesma. O presente documento tem como objetivo resolver esse problema de otimiza\c c\~ao implementando 
um conjunto de metaheur\'isticas incluindo estrat\'egias construtivas e por modifica\c c\~ao, e ainda realizando um estudo experimental a partir do resultado 
destas implemeta\c c\~oes
 \end{resumo}

\bigskip
\begin{palchaves}
Aloca\c c\~ao de testes, Metaheur\'isticas, Problemas de otimiza\c c\~ao.

\end{palchaves}


\vspace{8mm}

\begin{abstract}
The test-Assignment problem is a complex optimization problem, the main objetive of the problem is to find the best way to distribute a test within 
a room in order to minimize the probability of copying answers among the applicators. The document aims to solve this optimization problem implementing a
set of metaheuristics including constructive strategies and modification, and still and still carrying out an experimental study on the result
of these implements.

\end{abstract}

\bigskip
\begin{keywords}
Test-Assignment. Metaheuristics. Optimization problems.

\end{keywords}

 
\newpage
\section{Introdu\c{c}\~ao} 
 
A partir do memento que temos a aplica\c c\~ao de um teste com um numero x de concorrentes queremos eliminar ao m\'aximo a probabilidade de que possa ocorrer copia de respostas de outros concorrentes 
pois isso estar\'a afetando o resultado final dos melhores classificados, a partir disso que surge o problema de aloca\c c\~ao de de testes. 
A probabilidade e facilidade de que se ocorra esse poss\'ivel roubo de respostas pode se dar a partir de alguns fatores,
sendo eles por exemplo a proximidade entre as carteiras e o qu\~ao parecido e cada uma das provas a cada par de carteiras pode ser, a partir desses fatores pode ser 
definido uma nota a para a facilidade de que ocorra ou essa infra\c c\~ao de copia de respostas entre provas, ou seja o objetivo de o problema \'e minimizar ao m\'aximo
essa possibilidade. 

Para isso devemos come\c car modelando nosso problema, inicialmente podemos representa-lo como sendo um grafo, cada v\'ertice sendo uma carteira e cada arestas ir\'a conter
o um pesso que define a facilidade que ocorra a famosa cola, ou seja um as respostas de um teste sejam copiadas por outra pessoa que esteja fazendo o mesmo teste. Esse pesso como
abordado anteriormente ser\'a definido pela jun\c c\~ao do peso da distancia entre os pares de carteiras e de qu\~ao cada prova \'e parecida entre estes mesmos pares.

Este problema foi introduzido inicialmente Duives~\citep{duives:13}. para melhorar essa distribui\c c\~ao de testes aplicados na Universidade de Bologna. A partir de seu estudo inicial foi demonstrado que o 
problema pode ser classificado como NP-Dif\'icil, e o formulou o problema como um programa quadr\'atico bin\'ario n\~ao convexo.

\section{Abordagens tradicionais}

Inicialmente \'e foi implementado as abordagens tracinonais para que sej\'a validado se quais resultados poss\'iveis antes de se ter que aplicar heur\'isticas constrututivas. As abordagens iniciais ecolhidas foram \textbf{busca local randomizada}, \textbf{busca tabu} e  \textbf{reinicio aleat\'orio}. Todas essas abordagens tem como objetivo sempre fazer uma an\'alise a fim de encontrar os melhores vizinhos que no contexto do problema seriam os com provas mais diferentes como tamb\'em a distancia entre eles.

\section{ Heur\'isticas Constrututivas}

Em seguida foi implementa novas abordagens utilizando heur\'isticas constrututivas, as abordagens escolhidas foram \textbf{constru\c c\~ao repetida}, e utilizando \textbf{algoritmos semi-gulosos: guloso-k e guloso-$\alpha$}. Nestas abordagens \'e feito um ranqueamento entre as melhores op\c c\~oes encontradas a fim de que o programa n\~ao acabe descartando uma solu\c c\~ao melhor. 

\section{ An\'alises dos Resultados }

\section{ Mix de abordagens }
Ap\'os a implenta\c c\~a das abordagens anteriores nesta se\c c\~a ser\'a implementado algumas implementa\c c\~oes onde misturamos mais de uma abordagem a fim de que ap\'os a primeira abordagem encontrar uma solu\c c\~ao boa a segunda abordagem possa pegar a solu\c c\~a e melhora-l\'a ainda mais. Um exemplo onde est\'a mistura pode ser ben\'efica \'e na busca tabu visto que utiliza de uma solu\c c\~ao aleat\'oria para como seu estado incial, caso essa solu\c c\~ao inicial seja boa vinda de alguma outra abordagem a busca ter\'a a aportunidade de melhorar ainda mais este resultado.

\section{ Conclus\~ao }

~\\
\bibliographystyle{sbpo}
\bibliography{exemplo-latex}


\end{document}

